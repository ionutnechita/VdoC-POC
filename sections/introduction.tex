\setlength{\parindent}{0.8cm}

Observam in prezent o multitudine de dispozitive electronice care sunt echipate
cu procesoare(microprocesoare) puternice, fiabile si pregatite sa indeplineasca 
sarcinile pentru care au fost construite. Dispozitive fixe, mobile, 
speciale, critice, etc; acestea ne ajuta sa indeplinim sarcinile grele mult 
mai usor. Dar acest aspect creeaza o integrare pe orizontala, prin multitudinea 
de dispozitive. In timp ce integrarea pe verticala ramane in principal in zona 
software. Prin integrare ne referim la dezvoltare de solutii, atat hardware 
si software. Putem intelege de aici ca integrarea orizontala si verticala 
indeplinesc cu succes cerintele.

Remarcam multitudinea de dispozitive si optiuni software care sunt imbunatatite 
in fiecare zi, pentru a beneficia de cea mai buna experienta de utilizare.
Dar observam ca acest lucru nu e de ajuns si ca limitarile exista, intre 
cele doua integrari si chiar inauntrul uneia sau a altei integrari.

Spunem ca dispozitivele electronice sunt performante, productive, eficiente,
calitative si asigura un suport impresionant. Dar acestea folosesc procesoare
care indeplinesc anumite sarcini date de utilizator care in majoritatea cazurilor
ele sunt foarte comune sau prezinta o recurenta. Ceea ce arata ca procesoarele 
ori nu sunt utilizate indeajuns, ori nu sunt folosite optim.

Limitarile pe care le avem in integrarea pe orizontala pot fi reduse pe parcurs
prin imbunatatirea procesului arhitectural si de constructie a procesoarelor.
Vedem de la an la an acest lucru ca se intampla, dar acesta este unul lent,
din cauza mai multor factori, unul dintre ei este arhitectural.

Arhitectura procesoarelor este una simpla in acest moment, un procesor cu un
numar exact de nuclee si fire de executie, cu nivel de memorie cache limitat
si imbinat. De aici putem observa ca limitarile sunt prezente. Un punct de plecare
pentru a reduce limitarile ar fi procesoarele hibride sau cu mai multe componente, 
SoC(Sistem in Cip, System on a Chip). Dar acestea devin dupa o perioada limitate 
din cauza componentelor si functiilor care sunt strans legate.

Viitorul dispozitivelor este creat prin procesoare hibride, dar cu optimizare 
dinamica. Ceea ce observam ca nu se intampla. Optimizarea dinamica din procesoare
nu exista sau este una precoce.

Conceptul de VdoC(Dispozitiv vibrant pe un Cip) este creat pentru a optimiza
acest aspect limitativ prin optimizarea automata a tuturor resurselor din cadrul
unui dispozitiv vibrant integrat pe un cip. Termenul vibrant este folosit pentru 
a reprezenta gestionarea/optimizarea automata a resurselor si luarea deciziilor
autonome.

